\documentclass[/Users/zhangwusheng/Documents/satz von radon nikodym/satz von radon nikodym.tex]{subfiles}
\begin{document}
\section{Maß mit Dichten- Satz von Radon-Niodym}
    Es sei beliebiger Maßraum $(\omega, \mathcal{A}, \mu)$ und sei $f$ eine meßbare nichtnegative oder integrierbare Funktion
    auf $\omega$. 
    In diesem Kapital wollen wir uns das Verhalten dieses Integrals in Abhängigkeit von jedem Menge $A \in \mathcal{A}$ interessieren.

    \begin{theorem}\label{satz:4}
        Für jede nichtnegative Funktion $f$ wird durch 
        \[v(A) = \int_A f\,\mathrm{d}\mu\]
        ein Maß auf $A \in \mathcal{A}$ definiert.
    \end{theorem}

    \begin{proof}
        Es ist $v(\emptyset)=0$ und $v \geq 0$. Nun zeigen wir, $v$ abzählbare Aktivität folgt. Für jede Folge $(A_{n})_{n \in \mathbb{N}}$ 
        paarweise fremder Mengen aus $\mathcal{A}$ mit 
        $A = \cup_{n=1}^\infty A_{n}$ gilt 
        \[\mathbbm{1}_{A} f = \sum_{n=1}^\infty \mathbbm{1}_{A_n} f\]
        und aus Satz von der monotonen Konvergenz, folgt 
        \[v(A) = \sum_{n=1}^\infty v(A_n).\]
        Also ein Maß auf $\mathcal{A}$. 
    \end{proof}

    \begin{mdframed}[style=mdfexample]
        \begin{definition}
            Ist $f$ eine nichtnegative, $\mathcal{A}$-messbare Funktion auf $\Omega$, so heißt das durch \cref{satz:4} auf $\mathcal{A}$ 
            definierte Maß $v$ das \textit{Maß mit der Dichte $f$} bezüglich $\mu$. Es wird auch mit 
            \[v = f\mu\]
            bezeichnet.
        \end{definition}
    \end{mdframed}
    Über den Zusammenhängzwischen $\mu$ und $v$ wollen wir die folgende Eingenschften zeigen.

    \begin{theorem}\label{satz:5}
        Für je zwei $\mathcal{A}$-messbare Funktion $f, g$ gilt 
        \[f = g\, \text{$\mu$-fast überall}\, \Longrightarrow f\mu = g\mu.\]
        Ist $f$ oder $g \mu$-integrierbar, so gilt auch die Umkehrung.
    \end{theorem}

    Bevor wir diese Aussage beweisen, zurest müssen wir die folgende Lemma beweisen.
    
    \begin{lemma}\label{lem:6}
        Für jede nichtnegative $\mathcal{A}$-messbare Funktion $f$ auf $\Omega$, gilt 
        \[\int f\,\mathrm{d}\mu \Longleftrightarrow f=0\, \text{$\mu$-fast überall}.\]
    \end{lemma}

    \begin{proof}
        Wegen der Meßbarkeit von $f$ liegt die Menge 
        \[N = \{f \neq 0\} = \{f > 0\}\]
        in $\mathcal{A}$ und wir zeigen, 
        \[\int f\,\mathrm{d}\mu \Longleftrightarrow \mu(N)=0.\]
        Sei $\int f\,\mathrm{d}\mu = 0$. Sei die Folge $A_n = \{f \geq n^{-1}\}, n \in \mathbb{N}$ der Mengen liegt in $\mathcal{A}$,
        und es gilt $A_n \uparrow N$. Wegen $\mu(N) = \lim \mu(A_n)$, zeigen wir also $\mu(A_n) =0$ für alle $n$. Dann gilt $f \geq n^{-1} \mathbbm{1}_{A_n}$
        und somit 
        \[0 = \int f\,\mathrm{d}\mu \geq n^{-1}\mu(A_n) \geq 0,\]
        also gilt $\mu(N) =\lim \mu(A_n) =0$.\\
        Sei umgekehrt. Sei $\mu(N)=0$. Für jede Folge der Funktionen $u_n = n \mathbbm{1}_N, n \in \mathbb{N}$ und ist $\int u_n\,\mathrm{d}\mu =0$.
        Setzen wir also $g = \sup{u_n}$, so folgt $u_n \uparrow g$ und aus Satz von der monotonen Konvergenz folgt 
        $\int g\,\mathrm{d}\mu = \sup \int u_n\,\mathrm{d}\mu=0$. Schließlich ist $f \leq g$ und damit 
        \[0 \leq \int f\,\mathrm{d}\mu \leq \int g\,\mathrm{d}\mu \leq 0,\]
        also gilt $\int f\,\mathrm{d}\mu = 0$.  
    \end{proof}

    \textbf{Beweis \cref{satz:5}}
    \begin{proof}
        Mit $f$ und $g$ stimmen für jede Menge $A \in \mathcal{A}$ auch die Funktionen 
        $\mathbbm{1}_{A} f$ und $\mathbbm{1}_{A} g$ stimmen fast überall herein. Folglich gilt 
        \[\int_A f\,\mathrm{d}\mu = \int_A g\,\mathrm{d}\mu.\]
        für alle $A \in \mathcal{A}$, also gilt $f\mu = g\mu$.\\
        Zeigen wir die Umkehrung. Sei $f$ $\mu$-integrierbar und $f\mu = g\mu$. Wegen $g \geq 0$ und 
        $\int g\,\mathrm{d}\mu = \int f\,\mathrm{d}\mu < \infty$, ist $g$ auch $\mu$-integrierbar.\\ 
        Wir zeigen, dass die in $\mathcal{A}$ gelegte Menge
        \[N = \{f > g\}\]
        eine $\mu$-Nullmenge ist.\\
        Für alle $\omega \in N$ ist die Funktion $h(\omega) = f(\omega) - g(\omega)$ definiert und $h > 0$. 
        Also ist die Definition 
        \[h= \mathbbm{1}_{N} f - \mathbbm{1}_N g\]
        sinnvoll. Wegen $\mathbbm{1}_N f \leq f$ und $\mathbbm{1}_N g\leq g$ sind auch die Funktionen 
        $\mathbbm{1}_N f$ und $\mathbbm{1}_N g$ $\mu$-integrierbar. Wegen $f\mu = g\mu$ besitzen die beiden Funktionen
        dassselbe $\mu$-Integral, d.h.
        \[\int h\,\mathrm{d}\mu = \int_{N} f\,\mathrm{d}\mu - \int_N g\,\mathrm{d}\mu = 0.\]
        Heraus folgt $\mu(h) = 0$, wegen \cref{lem:6}, da $N = \{h > 0\}$ gilt.  
    \end{proof}
    Nun kommen wir zu dem angeküngiten Hauptproblem: Auf der Maßraum $(\Omega, \mathcal{A})$ seien zwei Maße $\mu$ und $v$ 
    gegeben. Wie man entscheiden kann, ob $v$ eine Dichte bezüglich $\mu$ besitzt. Notwendig hierfür ist die Bedingung, 
    dass jede $\mu$-Nullmenge $A \in \mathcal{A}$ auch eine $v$-Nullmenge ist.
    
    \begin{mdframed}[style=mdfexample]
        \begin{definition}
            Sei $(\Omega, \mathcal{A}, \mu)$ ein Maßraum und $v$ ein Maß auf $\mathcal{A}$. Dann heißt $v$ \textit{absolut stetig} 
            bezüglich $\mu$, kurz: $\mu$-stetig, wenn jede $\mu$-Nullmenge auf $\mathcal{A}$ auch eine $v$-Nullmenge ist. Man schreibt dafur $v \ll \mu$.
        \end{definition}
    \end{mdframed}
    Wir wollen sehen, wie der Begriff der Absolutstetigkeit von Maßen mit dem Begriff der Stetigkeit von Funktionen zusammenhängt. 
    Die nächste Behauptung läßt bereits eine Ähnlichkeit mit der Stetigkeit von Funktionen erahnen. 

    \begin{theorem}
        Gegeben seien endliche Maße $\mu$ und $v$ auf einem Maßraum $(\Omega, \mathcal{A})$, und $v$ ist genau dann $\mu$-stetig, wenn zu jeder Zahl $\varepsilon > 0$ exsistiert eine Zahl $\delta > 0$, dass 
        \[\mu(A)\leq \delta \Longrightarrow v(A) \leq \varepsilon,\]
        für alle $A \in \mathcal{A}$ gilt. 
    \end{theorem} 
    
    \begin{proof}
        Aus der Bedingung folgt $v(A) \leq \varepsilon$ für jedem Nullmenge $A \in \mathbb{A}$. Dann gilt $v(A)=0$ und somit ist $v$ ein $\mu$-stetiges Maß.\\
        Umgekehrt. Zeigen wir, dass $v$ kein $\mu$-stetiges Maß, wenn die Bedingung nicht erfüllt. Es gibt ein $\varepsilon > 0$ und eine Folge $(A_n)_{n \in \mathbb{N}}$ mit den Eingenschften
        \[\mu(A_n) \leq 2^{-n} \text{and}\, v(A_n) > \varepsilon.\]
        Setzen wir 
        \[A = \limsup_{n \to \infty} A_n = \bigcap_{n=1}^\infty \bigcup_{m=n}^\infty A_m,\]
        so für diese Menge aus $\mathcal{A}$ gilt 
        \[\mu(A)\leq \mu\bigl(\bigcup_{m=n}^\infty A_m\bigr) \leq \sum_{m=n}^\infty A_m \leq \sum_{m=n}^\infty 2^{-m} = 2^{-n+1}.\]
        also $\mu(A) =0$, wenn $n \rightarrow \infty$. Andererseits wegen der Endlichkeit von $v$ gilt  
        \[v(A) =v(\limsup_{n \to \infty} A_n) \geq \limsup_{n \to \infty} v(A_n) \geq \varepsilon > 0,\]
        gemäß der Lemma von Fatou.  Das Maß $v$ daher ist nicht $\mu$-stetig.
    \end{proof}

    Wie bereits angeküngit, ergibt sich die Beantwortung der wichtigen Frage.

    \begin{mdframed}[style=mdfexample]
        \begin{theorem}[Satz von Radon-Nikodym]\label{satz:6}
            Sei $(\Omega, \mathcal{A})$ ein Maßraum, und gegeben seien $\mu$ $\sigma$-endliche Maße auf $\mathcal{A}$ mit $v \ll \mu$. 
            Dann gibt es eine nichtnegative $\mathcal{A}$-messbare Funktion $f$ mit
            \[v(A) = \int_{A} f\,\mathrm{d}\mu\]
            für alle $A \in \mathcal{A}$.
            Die Funktion $f$ ist eindeutig bestimmt, d.h. Ist $\tilde{f}$ eine weitere Dichte von $v$ bezüglich $\mu$,
            so gilt $f = \tilde{f}$ $\mu$-fast überall, mit $\mu(\{f \neq \tilde{f}\}) = 0$.
        \end{theorem}
    \end{mdframed}

    \begin{remark}
        Man kann in \cref{satz:6} auf die Voraussetzung der $\sigma$-Endlichkeit von $v$ verzichten, allerdings ist $f$ dann eine nicht mehr reellwertige Funktion.  
    \end{remark}

    \begin{example}
        Sei $I=[0,1]$ und $\mathcal{A}$ die $\sigma$-Algebra aller Teilmengen in $I$, für welche entweder $A$ oder $A^{\mathrm{c}}$ abzählbar
        ist. Seien $v$ ein Lesbegue Maß auf $\mathcal{A}$ und $\mu$ das Zählmaß auf $\mathcal{A}$. Dann $\emptyset$ ist die einzige Nullmenge, und somit $v$ trivialerweise $\mu$-stetig. Aber kann 
        $v$ keine Dichte bezüglich $\mu$ besitzen. Aus $v=f \mu$ mit $f \in L^1(\mu)$ würde nämlich für welche $\omega \in I$ folgen:
        \[0= v(\{\omega\})= \int_{\{\omega\}} f\,\mathrm{d}\mu = f(\omega) \mu(\{\omega\}) = f(\omega) \cdot 1 = f(\omega),\]
        d.h. $f$ genau 0 ist. Jedoch folgt 
        \[1= v(I) = \int_{I} f\,\mathrm{d}\mu = \int_I 0\,\mathrm{d}\mu = 0.\]
        Widerspruch!
    \end{example}

    \begin{remark}
        Mann kann in \cref{satz:6} die Voraussetzung der absolute stetigkeit auf $v$ bezüglich $\mu$ nicht verzichten.
    \end{remark}

    \begin{example}
        Sei $I=[0,1]$ und $\mathcal{A}$ die $\sigma$-Algebra aller Teilmengen in $I$. Sei $v$ ein Lesbegue Maß. Betrachten wir das Maß $\mu$ mit 
        \[\mu(E) = \#(E \cap \mathbb{Q})\]
        für alle $E \in I$, d.h. $\mu$ nimmt die Anzahl der rationalen Elementen in Menge $E$ und bzw. $\mu(E)=0$ wie sonst. Wegen $\mathbb{Q}=\{r_n\}_{n=1}^\infty$, erhalten wir
        \[[0,1]=([0,1] \cap (I \setminus \mathbb{Q})) \cup \bigcup_{n=1}^\infty r_{n}.\]
        Dann folgt $\mu$ ein nichnegatives $\sigma$-endliches Maß. Jedoch erhalten wir  
        \[\mu(I \setminus \mathbb{Q})= 0,\]
        und 
        \[v(I \setminus \mathbb{Q}) = 1,\]
        und somit ist $v$ nicht $\mu$-stetig. Wäre $v= f\mu$ mit $f \in L^1(\mu)$, dann erhalten wir
        \[0= v(\{\omega\})=\int_{\{\omega\}} f\,\mathrm{d}\mu = f(\omega)\mu(\{\omega\}) =f(\omega),\]
        also $f(\omega)=0$ für aller $\omega \in \mathbb{Q}$. Dann aber Wäre $v=0$, was falsch ist. 
    \end{example}

    \textbf{Beweis \cref{satz:6}:}
    \begin{proof}
        Wir unterscheiden den zwei Fälle.\\
        \textit{Fall 1}. Wir beschränken uns auf den Fall endlicher Maße $\mu$ und $v$.\\
        \textit{Schritt 1}. Zuerst betrachten wir 
        das Funktionensystem aller $\mathcal{A}$-messbaren Funktionen $g$ mit 
        \[\mathfrak{G} = \Bigl\{g \geq 0 \Bigl|\, \forall A \in \mathcal{A}, \int g\,\mathrm{d}\mu \leq v(A)\Bigr\}.\]
        Da $g \equiv 0$ zu $\mathfrak{G}$ gehört, so dass $\mathfrak{G}$ nicht leer ist.\\
        \textit{Schritt 2}. Wir vermuten, dass $\mathfrak{G}$ maximales Element enthält und dieses die geforderten Eigenschaften erfüllt.\\
        Setzen wir nun
        \[\alpha = \sup_{g \in \mathfrak{G}} \int g\,\mathrm{d}\mu.\]
        Man beachten, dass das Supremum endlich ist, weil $v$ endlich ist, d.h. 
        \[\alpha = \sup_{g \in \mathfrak{G}} \int_{\Omega} g\,\mathrm{d}\mu \leq v(\Omega) < \infty.\]
        Es gibt eine Folge der $\mathcal{A}$-messbaren Funktionen $(\tilde{g}_n)_{n \in \mathbb{N}} \in \mathfrak{G}$ mit 
        \[\lim_{n \to \infty} \int_{\Omega} \tilde{g}_n\,\mathrm{d}\mu = \sup_{g \in \mathfrak{G}} \int_{\Omega} g\,\mathrm{d}\mu.\]
        Nun setzen wir auch eine Folge 
        \[g_n = \max_{1 \geq j \geq n}{(\tilde{g}_j)},\]
        für die wegen $\tilde{g}_n \leq g_n$ die Relation 
        \[\int \tilde{g}_n\,\mathrm{d}\mu \leq \int g_n\,\mathrm{d}\mu\]
        gelten. 
        Nun zeigen wir, dass mit zwei Funktionen $g_1$ und $g_2$ auch deren Maximum zu $\mathfrak{G}$ gehört. Hier setzen wir die Menge $A_1 = \{g_1 \geq g_2\}$ und 
        $A_1 = {A_1}^{\mathrm{c}}$. Dann ergibt sich für jedes $A \in \mathcal{A}$ zu die folgende Umgleichung 
        \begin{align*} 
            \int_A \max{(g_1, g_2)}\,\mathrm{d}\mu &= \int_{A \cap A_1} \max{(g_1, g_2)}\,\mathrm{d}\mu + \int_{A \cap A_2} \max{(g_1, g_2)}\,\mathrm{d}\mu\\
            &= \int_{A \cap A_1} g_1\,\mathrm{d}\mu + \int_{A \cap A_2} g_2\,\mathrm{d}\mu\\
            &\leq v(A \cap A_1) + v(A \cap A_2)\\
            &=v(A)
        \end{align*}
        Mit Induktion beweisen wir, dass $g_n$ in $\mathfrak{G}$ gehört.\\
        \textit{Schritt 3}. Dann zeigen wir, dass die Funktion $f \in \mathfrak{G}$ existiert mit   
        \[\int_{\omega} f\,\mathrm{d}\mu = \lim_{n \to \infty}\int_{\Omega} g_n\,\mathrm{d}\mu\]
        Da $(g_n)_{n \in \mathbb{N}}$ die monoton wachsende Folge und  
        \[g_n \uparrow f\]
        für eine gewisse messbare Funktion $f \geq 0$. Wir zeigen, dass $f \in \mathfrak{G}$ gilt. Für beliebiges $A \in \mathcal{A}$, wegen $\mathbbm{1}_{A} g_n \uparrow \mathbbm{1}_A f$ mit dem Satz von der
        monotonen Konvergenz folgt 
        \[\int_A f\,\mathrm{d}\mu = \lim_{n \to \infty} g_n\,\mathrm{d}\mu.\]
        Wegen $(g_n)_{n \in \mathbb{N}}$ in $\mathfrak{G}$ enthält, ist das Integral von $f$ nach oben beschränkt, d.h. 
        \[\int_A f\,\mathrm{d}\mu \geq v(A) < \infty,\]
        und somit $f$ in $\mathfrak{G}$ gehört\\
        \textit{Schritt 4}. Dann zeigen wir, dass 
        \[\int_{\Omega} f\,\mathrm{d}\mu = \int_{\omega} g\,\mathrm{d}\mu.\]
        Wegen $f \in \mathfrak{G}$ gilt  
        \[\int_{\Omega} f\,\mathrm{d}\mu \leq \int_{\Omega} g\,\mathrm{d}\mu.\]
        Andererseits wegen $g_n \geq \tilde{g}_n$, folgt 
        \[\int_{\Omega} f\,\mathrm{d}\mu = \lim_{n \to \infty} g_n\,\mathrm{d}\mu \geq \lim_{n \to \infty} \tilde{g}_n\,\mathrm{d}\mu = \sup_{g \in \mathfrak{G}} \int_{\Omega} g\,\mathrm{d}\mu,\]
        d.h. 
        \[\int_{\Omega} f\,\mathrm{d}\mu \geq \sup_{g \in \mathfrak{G}} \int_{\Omega} g\,\mathrm{d}\mu.\]
        Folglich wegen 
        \[\int_{\omega} f\,\mathrm{d}\mu = \int_{\omega} g\,\mathrm{d}\mu\]
        gilt das Funktionensystem $\mathfrak{G}$ ein maximales Element enthält.\\
        \textit{Schritt 5}. Jetzt zeigen wir, dass $v(A) = \int_{A} f\,\mathrm{d}\mu$ für alle $A \in \mathcal{A}$ gilt. Wir nehmen an, dass $v(A) \geq \int_{A} f\,\mathrm{d}\mu$ ist  
        und somit  
        \[\tau(A) = v(A) - \int_A f\,\mathrm{d}\mu, \quad \forall A \in \mathcal{A}\]
        ein endliches Maß existiert. Zu zeigen ist $\tau = 0$.\\ 
        Angenommen, es gilt $\tau(\Omega) > 0$. Da $\mu$ als endlich vorausgesetzt wurde, finden wir ein $\underline{\varepsilon > 0}$ mit
        \[\tau(\Omega) > \varepsilon \mu(\Omega).\]
        Sei $(\Omega^{+}, \Omega^{-})$ eine Hahn-Zerlegung des signierten Maß $\tau(\Omega) - \varepsilon \mu(\Omega)$. Dann erhalten wir 
        \[\tau(A) \geq \tau(A \cap \Omega^{+}) \geq \varepsilon \mu(A \cap \Omega^{+})\]
        für alle $A \in \mathcal{A}$. Dann existiert eine Funktion $f_0 = f + \varepsilon \mu(\Omega^{+})$ und daher bei beliebigem $A \in \mathcal{A}$ gilt 
        \begin{align*}
            \int_A f_0\,\mathrm{d}\mu &= \int_A f\,\mathrm{d}\mu + \varepsilon \mu(A \cap \Omega^{+})\\
            &\leq \int_A f\,\mathrm{d}\mu + \tau(A \cap \Omega^{+})\\
            &\leq \int_A f\,\mathrm{d}\tau(A)\\
            &=v(A).
        \end{align*}
        Dann liegt die Funktion $f_0$ auch in $\mathfrak{G}$. Wegen $\int f\,\mathrm{d}\mu = \alpha$ und der Definition der $\alpha$ muss auch 
        \[\int f_0\,\mathrm{d}\mu = \alpha,\]
        woraus 
        \[\mu(\Omega^{+}) =0\]
        sein. Wegen $v \ll \mu$ impliziert $v(\Omega^{+}) =0$ und foglich $\tau(\Omega^{+}) =0$. \footnote{Wir merken an, dass dies die einzige Stelle im Beweis ist, an der die Absolutstetigkeit des Maßes $v$ 
        bezüglich $\mu$ benutzt wird.} Deshalb erhatlen wir 
        \begin{align*}
            \tau(\Omega) - \varepsilon \mu(\Omega) &= (\tau - \varepsilon)(\Omega^{+}) + (\tau - \varepsilon)(\Omega^-)\\
            &= (\tau - \varepsilon)(\Omega^{-}) \leq 0
        \end{align*}
        Jedoch ist es Widerspruch an Definition der $\varepsilon$. 
        \textit{Schritt 6}. Es ist noch die Einzigkeit der Radon-Nikodym-Ableitung nachzuweisen. Seien $f$ und $g$ nichtnegative $\mathcal{A}$-messbare Funktionen mit 
        \[\int_A f\,\mathrm{d}\mu = \int_A g\,\mathrm{d}\mu < \infty\]
        für alle $A \in \mathcal{A}$. Es ist zu zeigen, dass hieraus $f = g$ $mu$-fast überall ist. Setzen wir $A= \{f \geq g\}$ und nun zeigen wir, dass $\mu(A) =0$. Wegen $f$ und $g$ die $\mathcal{A}$-messbare Funktionen, 
        erhalten wir $\mathbbm{1}_A (f-g)$ auch eine $\mathcal{A}$-messbare Funktion, dann gilt 
        \[\int \mathbbm{1}_A (f-g)\,\mathrm{d}\mu = \int_A (f-g)\,\mathrm{d}\mu = \int_A f\,\mathrm{d}\mu - \int_A g\,\mathrm{d}\mu = 0.\]
        Mit der \cref{lem:6} folgt $\mu(A) =0$ und dann ist $f \geq g$ $\mu$-f.ü. Analog zeigt man $f \leq g$ $\mu$-f.ü.\\
        \textit{Fall 2}. Wir betrachten, dass $\mu$ und $v$ $\sigma$-endlich sind.\\
        \textit{Schritt 1}. Wir möchten eine Doppelfolge $(B_n)_{n \in \mathbb{N}} = (C_l \cap D_m)_{l, m \in \mathbb{N}}$ mit paarweise disjunkter Mengen zu konstruieren.\\
        Zurest betrachten wir $\mu$ und $v$ $\sigma$-endliche Maße auf $\mathcal{A}$, und seien die Folgen $(C_n)_{n \in \mathbb{N}}$ und $(D_n)_{n \in \mathbb{N}}$ paarweise disjunkter Mengen aus $\mathcal{A}$ mit 
        $\Omega = \bigcup_{n=1}^\infty C_n = \bigcup_{n=1}^\infty D_n$ und $\mu(C_n) < \infty$ und $v(D_n) < \infty$. Dann gilt es  
        \[C_n = C_n \cap \Omega = C_n \cap (\bigcup_{n=1}^\infty D_n) = \bigcup_{n=1}^\infty C_n \cap D_n,\]
        und somit 
        \[\Omega = \bigcup_{n=1}^\infty C_n = \bigcup_{n=1}^\infty (C_n \cap D_n).\]
        Sezten wir $(B_n)_{n \in \mathbb{N}}$ die Aufzählung der Elementen auf dolppelfolge $(C_l \cap D_m)_{l, m \in \mathbb{N}}$. Dann ist $(B_n)_{n \in \mathbb{N}}$ auch eine Folge mit paarweise disjunkter Elementen aus $\mathcal{A}$.\\
        \textit{Schritt 2}. Setzen wir $\mu_n, v_n$ nichtnegative Maße $\mathcal{A} \rightarrow \mathbb{R}^{+}$ für jedes $n \in \mathbb{N}$ mit $\mu_n(A) = \mu(A \cap B_n) \leq \mu(B_n) < \infty$ und $v_n(A) = v(A \cap B_n) \leq v(B_n) < \infty$, so dass $\mu_n$ und $v_n$ endliche sind.
        Wir zeigen, dass $v_n \ll \mu_n$ ist. Ist $v_n(A) =0$, so ist $v(A \cap B_n) =0$, und wegen $v \ll \mu$, folgt $\mu(A \cap B_n) =0$ und $\mu_n(A)=0$, und damit $v_n \ll \mu_n$.\\
        \textit{Schritt 3}. Ist $A \in \mathcal{A}$ eine $\mu$-messbare Menge und $B_n \in \mathcal{A}$, und damit $A \cap B_n$ auch in $\mathcal{A}$ enthält. Folglich ist $A \cap B_n$ $\mu$-messbar und auch $\mu_n$-messbar.
        Jetzt erfüllt es die Voraussetzung der \cref{satz:6}, und existiert es eine Folge der nichtnegativen $\mu_n$-messbaren Funktionen $(f_n)_{n \in \mathbb{N}}$, so dass 
        \[v_n = \int_A f_n\,\mathrm{d}\mu_n\]
        für alle $\mu_n$-messbare Menge $A \in \mathcal{A}$. Definieren wir $f$ eine nichtnegative $\mu$-messbare Funktion durch $f(x) = \sum_{n=1}^\infty f_n(x)\mathbbm{1}_{B_n}$ für alle $x \in B_n$. Dann wir zeigen, dass 
        \[v(A) = \int_A f\,\mathrm{d}\mu\]
        für jedem $A \in \mathcal{A}$. Sei $A \in \mathcal{A}$ mit $v(A) < \infty$. Wegen $B_n$ paarweise disjunkt ist, folgt $A \cap B_n$ paarweise disjunkt. Dann gilt 
        \begin{align*}
            v(A) &= v(A \cap \Omega)= v\bigl(A \cap \bigcup_{n \in \mathbb{N}} B_n \bigr) = v\bigl(\bigcup_{n \in \mathbb{N}} (A \cap B_n)\bigr)\\
            &= \sum_{n \in \mathbb{N}} v(A \cap B_n) = \sum_{n \in \mathbb{N}} v_n(A)\\
            &= \sum_{n \in \mathbb{N}} \int_A f_n\,\mathrm{d}\mu_n = \sum_{n \in \mathbb{N}} \int_{A \cap B_n} f_n\,\mathrm{d}\mu \quad \bigstar\\
            &= \int_A \bigl(\sum_{n=1}^\infty f_n \cdot \mathbbm{1}_{B_n}\bigl)\,\mathrm{d}\mu\\
            &= \int_A f\,\mathrm{d}\mu
        \end{align*}
        \textit{Schritt 5}. Wir müssen die Gleichung $\bigstar$ beweisen, d.h. 
        \[\forall A \in \mathcal{A}, \int_A f\,\mathrm{d}\mu_n = \int_{A \cap B_n} f\,\mathrm{d}\mu.\]
        Setzen wir eine Elementarfunktion $s = \sum_{n=1}^m \alpha_n \cdot \mathbbm{1}_{A_n}$ für beliebige $A_n \in \mathcal{A}$ und 
        \[\int_A f\,\mathrm{d}\mu_n = \sup \bigl\{\int_A s\,\mathrm{d}\mu_n \bigl|\, 0 \leq f \leq f\bigr\}\]
        \[\int_{A \cap B_n} f\,\mathrm{d}\mu =\sup \Bigl\{\int_{A \cap B_n} s\,\mathrm{d}\mu\bigl|\, 0 \leq s \leq f\Bigr\}\]
        für alle $A \in \mathcal{A}$. 
        Sei jede $A_n \in \mathcal{A}$, und somit gilt 
        \begin{align*} 
            \int_A s\,\mathrm{d}\mu_n &= \sum_{n=1}^m \alpha_n \cdot \mu_n(A \cap A_n)\\
            &= \sum_{n=1}^m \alpha_n \cdot \mu(A \cap A_n \cap B_n)\\
            &= \int_{A \cap B_n} s\,\mathrm{d}\mu
        \end{align*}
        Dann folgt 
        \[\int_A f\,\mathrm{d}\mu_n = \int_{A \cap B_n} f\,\mathrm{d}\mu.\]
        \textit{Schritt 6}. Wir müssen noch die Eindeutigkeit nachweisen. Es gibt $g$ eine nichtnegative, $\mathcal{A}$-messbare Funktion mit 
        \[\forall A \in \mathcal{A}, v(A) = \int_A g\,\mathrm{d}\mu,\]
        und somit gilt
        \begin{align*}
            \int_A f\,\mathrm{d}\mu_n &= \int_{A \cap B_n} f\,\mathrm{d}\mu\\
            &= v(A \cap B_n)\\
            &= \int_{A \cap B_n} g\,\mathrm{d}\mu\\
            &= \int_A g\,\mathrm{d}\mu_m.
        \end{align*}
        Es folgt $f = g$ $\mu_n$-fast überall, wegen $\mu_n$ und $v_n$ endlich sind, wie wir in \textit{Fall 1} bewiesen. 
        Es gibt eine Menge $A = \{f \neq g\}$ und zeigen wir, dass $\mu(A) = 0$ ist. Wegen $\mu_n(A) =0$, erhalten wir 
        \begin{align*} 
            \mu(A) &= \mu(A \cap \Omega)= \mu \bigl(A \cap \bigl(\bigcup_{n \in \mathbb{N}} B_n\bigr))\bigr)= \mu \bigl(\bigcup_{n \in \mathbb{N}} (A \cap B_n)\bigr)\\
            &= \sum_{n \in \mathbb{N}} \mu_n(A)\\
            &= \sum_{n \in \mathbb{N}} 0\\
                   &=0,        
        \end{align*}
        so dass $f=g$ $\mu$-f.ü. gilt. 
    \end{proof}
    Sehr mühsam!
    \begin{theorem}[Zerlegungssatz von Lesbegue]\label{satz:7}
        Sind $\mu$ und $v$ $\sigma$-endliche Maße auf $(\Omega, \mathcal{A})$, so läßt sich genau eine Weise in der Form 
        $v = v_1 + v_2$ mit Maßen $v_1$ und $v_2$ auf $\mathcal{A}$ darstellen, wobei $v_1 \ll \mu$ und $v_2 \bot \mu$ gilt.
      \end{theorem}      
\end{document}