\documentclass[/Users/zhangwusheng/Documents/satz von radon nikodym/satz von radon nikodym.tex]{subfiles}
\begin{document}
\section{Signierte Maße}
    In diesem Kapitel wollen wir uns mit $\sigma$-additiven Funktionen beschaftigen, die nicht notwendig nichtnegativ sind. 
    Als Motivation betrachten wir ein Integral
    \[v(A)=\int_{A} f\,\mathrm{d}\mu.\]
    Wir haben schon früher festgestellt, dass dadurch eine $\sigma$-additive Funktion definiert ist, wenn das Integral von $f$ existiert. 
    Dieses Integral können wir zerlegen:
    \[v(A)=\int_{A} f_{+}\,\mathrm{d}\mu - \int_{A} f_{-}\,\mathrm{d}\mu.\]
    Wenn wir $v_1(A) =\int_{A}f_{+}\,\mathrm{d}\mu$ und $v_2(A) =\int_{A}f_{-}\,\mathrm{d}\mu$ setzen, dann haben wir dadurch eine Darstellung
    \[v(A) = v_1(A) - v_2(A)\]
    von $v$ als Differenz zweier Maße gefunden.
    \begin{mdframed}[style=mdfexample]
        \begin{definition}
            Es sei ein Maßraum $(X, \mathcal{A}, \mu)$, und eine Funktion $\mu: \mathcal{A} \rightarrow [-\infty, \infty]$ heißt \textit{signierte Maßfunktion} (oder signiertes Maß), wenn:
            \begin{enumerate}
            \setlength\itemsep{-1em}
                \item $v(\emptyset) =0;$\\
                \item ihr Wertebereich entweder $(-\infty, \infty]$ oder $[−\infty, \infty)$ ist, d.h. es können nicht sowohl $\infty$ und $−\infty$ als Werte auftreten;\\
                \item Falls $A_{j}, j \in \mathbb{N}$, eine Folge von disjunkten Mengen in $\mathcal{A}$ ist, dann gilt 
                \[\mu \Bigl(\bigcup_{j=1}^\infty A_{j}\Bigr) =\sum_{j=1}^\infty A_{j}.\]
            \end{enumerate}
        \end{definition}

        \begin{definition}
            Sei $\mu$ ein signiertes Maß auf dem messbaren Raum $(\Omega, \mathcal{A})$. Eine Menge $A \in \mathcal{A}$ heißt
            \begin{enumerate}
            \setlength\itemsep{-1em}
                \item positiv, falls $\mu(B) \geq 0$ für alle $B \in \mathcal{A}, B \subseteq A$ gilt;\\
                \item negativ, falls $\mu(B) \leq 0$ für alle $B \in \mathcal{A}, B \subseteq A$ gilt.
            \end{enumerate}
        \end{definition}
    \end{mdframed}
     
    \begin{mdframed}[style=mdfexample]
        \begin{lemma}\label{lem:1}
            Es sei $(\Omega, \mathcal{A})$ ein Maßraum, und $\mu$ ein signiertes Maß. Falls $A_j, j \in \mathbb{N}$, eine wachsende Folge von Mengen aus $\mathcal{A}$ ist, 
            d.h. für alle $j \in \mathbb{N}$ gilt $A_j \subseteq A_{j+1}$, dann gilt
            \[\mu \Bigl(\bigcup_{j=1}^\infty A_j\Bigr) =\lim_{j \to \infty} \mu(A_j)\]
            Falls $A_j, j \in \mathbb{N}$, eine fallende Folge von Mengen aus $\mathcal{A}$ ist, d.h. für alle $j \in \mathbb{N}$ gilt $A_{j+1} \subseteq A_{j}$, dann gilt
            \[\mu \Bigl(\bigcap_{j=1}^\infty A_j\Bigr)=\lim_{j \to \infty} \mu(A_j).\]
        \end{lemma}
    \end{mdframed}
    \begin{proof}
        Trivial, wie im Fall von positiven Maßen. 
    \end{proof}

    \begin{mdframed}[style=mdfexample]
        \begin{lemma}\label{lem:2}
            Es sei $(\Omega, \mathcal{A})$ ein Maßraum, und $\mu$ ein signiertes Maß. Es gibt ein Folge positiver Mengen $E_{n}$ aus $\mathcal{A}$. 
            Dann $\cup_{j=1}^\infty A_j$ eine positive Menge.
        \end{lemma}
    \end{mdframed}
    \begin{proof}
        Es sei $B \in \mathcal{A}$ eine Teilmenge von $\cup_{j=1}^\infty A_j$. Die Menge
        \[B \cap \Bigl(A_n \setminus \bigcup_{i=1}^{n-1} A_i \Bigr)\]
        ist disjunkte positive Mengen, weil sie Teilmenge von $A_n$ ist. Dann gilt
        \[\mu(B)=\sum_{j=1}^\infty \mu \Biggl(B \cap \Biggl(A_n \setminus \bigcup_{i=1}^{n-1} A_i \Bigr)\Biggr) \geq 0.\]
    \end{proof}

    \begin{mdframed}[style=mdfexample]
        \begin{lemma}\label{lem:4}
            Es sei $(\Omega, \mathcal{A})$ ein Maßraum, und $\mu$ ein signiertes Maß.
            Es gibt eine positive Menge $P$ und eine negative Menge $N$, so dass $P \cap N = \emptyset$.
        \end{lemma}
    \end{mdframed}
    \begin{proof}
        Wir zeigen, dass $P \cap N$ positiv noch negativ ist. Wir wählen eine Menge $F$, die als Teilmenge von $E \setminus F$ ist, 
        daher sind $F \subseteq P$ und $F \subseteq N$. $F \subseteq P \Longrightarrow \mu(F) \geq 0$ und $F \subseteq N \Longrightarrow \mu(F) \leq 0$. 
        Dann gilt $\mu(F) =0$, so dass $P \cap N =\emptyset$.
    \end{proof}
    Der zentrale Satz dieses Kapitels ist der
    \begin{mdframed}[style=mdfexample]
        \begin{theorem}[Zerlegung von Hahn]\label{satz:1}
            Es sei $(\Omega, \mathcal{A}, \mu)$ ein Maßraum mit einem endlichen signierten Maß. Dann existieren eine positive Mengen $\Omega^{+}$ und eine negative Menge $\Omega^{-}$, 
            mit $\Omega= \Omega^{+} \cup \Omega^{-}$ und $\Omega^{+} \cap \Omega^{-} =\emptyset$, so dass 
            \begin{enumerate}[label=(\alph*)]
                \item $\mu(A) \geq 0$, für alle $A \in (\Omega^{+} \cup \mathcal{A})$;
                \item $\mu(B) \geq 0$, für alle $B \in (\Omega^{-} \cup \mathcal{A})$.
            \end{enumerate}
        \end{theorem}
    \end{mdframed}
    Bevor wir diese Aussage beweisen, müssen wir zurest an die folgende Lemma beweisen, die als die wichtigste Lemma im Maßtheorie gezeigt ist. 
    \begin{mdframed}[style=mdfexample]
        \begin{lemma}\label{lem:5}
        Sei $\mu$ ein endliches signiertes Maß auf $(\Omega, \mathcal{A})$. Dann gibt eine Menge $\Omega_0 \in \mathbb{A}$ mit beiden 
        Eigenschaften
        \begin{enumerate}[label=(\alph*)]
            \item $\mu(\Omega_0) \geq \mu(\Omega) \quad \Omega_0 \subset \Omega$;
            \item $\mu(A) \geq 0 \quad \forall A \in (\Omega_0 \cap \mathcal{A})$. 
        \end{enumerate}
        \end{lemma}
    \end{mdframed}
    \begin{proof}
        Wir zeigen, dass zu jedem $\varepsilon >0$ gibt es eine Menge $\Omega_{\varepsilon} \in \mathcal{A}$
        mit beiden Eigenschaften:
        \begin{enumerate}[label=(\roman*)]
            \item $\mu(\Omega_{\varepsilon}) \geq \mu(\Omega) \quad \Omega_{\varepsilon} \subset \Omega$;
            \item $\mu(A) > -\varepsilon, \quad \forall A \in (\Omega_{\varepsilon} \cap \mathcal{A})$.
        \end{enumerate} 
        1. Wenn $\mu(\Omega) \leq 0$ ist, dann erfüllt $\Omega_{\varepsilon} = \emptyset$ diese Bedingungen.\\ 
        2. Offenbar kann dabei $\mu(\Omega) > 0$ vorausgesetzt werden. Wenn $\mu(A) > -\varepsilon$ für alle $A \in (\Omega\, \cap\, \mathcal{A})$ gilt, 
        dann erfüllt $\Omega_{\varepsilon} = \Omega$ beiden bedingungen, und das gesuchte $\Omega_{\varepsilon}$ gefunden.\\
        3. Also betrachten wir den Fall, dass eine Menge $A_1 \in (\Omega \cap \mathcal{A})$ mit $\mu(A_1) \leq - \varepsilon$ existiert.
        Da $\Omega$ die disjunkte Vereinigung von $A_1$ und $\Omega \setminus A_1$ ist. Aus der Definition von $\mu$ folgt, 
        \[\mu(\Omega \setminus A_1) = \mu(\Omega) - \mu(A_1) \geq \mu(\Omega) + \varepsilon > \mu(\Omega).\]
        Gilt daher $\mu(A) > - \varepsilon$ für alle $A \in (\Omega \setminus A_1) \cap \mathcal{A}$. Dann kann wir $\Omega_{\varepsilon} = \Omega \setminus A_1$ 
        gesetzt werden, und wir sind fertig.\\
        4. Ansonsten gibt es ein $A_2 \in (\Omega \setminus A_1) \cap \mathcal{A}$ und $\mu(A_2) \leq -\varepsilon$. Da $A_1 \cap A_2 = \emptyset$ erhalten wir
        \[\mu(\Omega \setminus (A_1 \cup A_2))= \mu(\Omega) - (\mu(A_1) + \mu(A_2)) \geq \mu(\Omega) + 2\varepsilon > \mu(\Omega),\]
        und gilt daher $\mu(A) > - \varepsilon$ für alle $A \in (\Omega \setminus (A_1 \cup A_2)) \cap \mathcal{A}$. Dann kann wir $\Omega_{\varepsilon} = \Omega \setminus 
        (A_1 \cup A_2)$ gesetzt werden.\\ 
        5. Wenn nicht, käme man so fortfahrend nach endlich vielen Schritten nicht zum Ziel, so ergäbe sich eine Folge $(A_n)$ paarweise fremder Mengen aus $\mathcal{A}$ mit 
        \[\mu(\Omega \setminus (A_1 \cup A_2 \cup \ldots \cup A_n)) > \mu(\Omega), \text{und}\, \mu(A_n) \leq - \varepsilon\]
        für alle $n \in \mathbb{N}$. Die Mengen $A_1, A_2, \ldots$ sind disjunkt. Für $A= \cup_{n=1}^\infty A_n$ folgt
        \[\mu(A)= \sum_{n=1}^\infty \mu(A_n) \leq \sum_{n=1}^\infty -\varepsilon = -\infty\]
        für alle $n \in \mathbb{N}$ und damit die Divergenz der Reihe $\sum_{n=1}^\infty \mu(A_n)$ liefern ein Widerspruch zu $\mu(A) \in \mathbb{R}$. 
        Damit ist gezeigt, dass ein $\Omega_{\varepsilon} \in \mathcal{A}$ existiert, für beiden Bedingungen (i), (ii) erfüllt ist.\\
        6. Um jetzt $\Omega_0$ zu finden. Sei $\Omega = \Omega_1$ als Grundmenge. Betrachten wir die Mengen $\Omega_{\varepsilon}$ mit beiden Eigenschaften (i) und (ii) für $\varepsilon = \frac{1}{n}$ und $n \in \mathbb{N}$.
        Wegen (i) folgt $\Omega_{1/(n+1)} \subset \Omega_{1/n}$ und $\mu(\Omega_{1/(n+1)}) \geq \mu(\Omega_{1/n})$ für alle $n \in \mathbb{N}$. Sind nämlich $\Omega_1 \supset \Omega_{1/2} \supset \Omega_{1/3} \supset
        \ldots \supset \Omega_{1/n}$ bereits konstruiert. Sei $\Omega_0 = \cap_{n=1}^\infty \Omega_n$. Für $A \in (\Omega_0 \cap \mathcal{A})$ gilt dann $\mu(A) > \frac{1}{n}$ für alle $n \in \mathbb{N}$ wegen (ii), also $\mu(A) \geq 0$, womit (b) gezeigt ist.
        Sei $E_{n} =\Omega_{1/n} \setminus \Omega_{1/(n+1)}$. Wegen $\Omega_{1/(n+1)} \subset \Omega_{1/n}$, ist $\Omega_{1/n}$ die disjunkte Vereinigung von $E_n$ und $\Omega_{1/(n+1)}$, so dass $\mu(\Omega_{1/(n+1)}) + \mu(E_n) = \mu(\Omega_{1/n})$ folgt. 
        Wegen $\mu(\Omega_{1/(n+1)}) \geq \mu(\Omega_{1/n})$, erhalten wir $\mu(E_n) \leq 0$. Folglich sind die Mengen $\Omega_{0}, E_1, E_2, \ldots$ disjunkt und ihre Vereinigung ist $\Omega$. Daher folgt $\mu(\Omega) = \mu(\Omega_0) + \sum_{n=1}^\infty \mu(E_n)$. 
        Wegen $\mu(E_n) \leq 0$ erhalten wir $\mu(\Omega) \leq \mu(\Omega_0)$. Da auch $\Omega_0 \subset \Omega$ gilt, ist (a) gezeigt. 
    \end{proof}

    \textbf{Beweis \cref{satz:1}}:
    \begin{proof}
        Wir nehmen an, dass der Werterbereich von $\mu$ ist $[-\infty, \infty)$, und wenn der Wertebereich von $\mu$ ist $[-\infty, \infty)$, nehmen wir $-\mu$.
        Sei $\alpha = \sup \{\mu(A)|\,A \subseteq \Omega\}$. Dann existiert $(B_n)_{n \in \mathbb{N}}$ eine Folge mit $\alpha = \lim_{n \to \infty} \mu(B_n)$.\\
        \textit{Schritt 1}.
        Aus \cref{lem:5} folgt die Existenz einer Folge $C_n \in \mathcal{A}$ mit $C_n \subset B_n$ und $\mu(C_n) \geq \mu(B_n)$. 
        Sei $D_n = C_n \setminus \cup_{i=1}^\infty C_i$. Die Mengen sind disjunkt. Sei $\Omega^{+} = \cup_{n=1}^\infty D_n = \cup_{n=1}^\infty C_n$. 
        Sei $A \in \mathcal{A}$ und $A \subset \Omega^{+}$. Setze $A_n = A \cap D_n$. Wegen $A_n \subset D_n \subset C_n$, aus \cref{lem:5} folgt $\mu(A_n) \geq 0$. 
        Da $A$ disjunkte Vereinigung der $A_n$ ist, folgt 
        \[\mu(A)= \sum_{n=1}^\infty \mu(A_n) \geq 0.\]
        Damit ist (a) gezeigt.\\
        \textit{Schritt 2}. Wir zeigen, dass $\mu(\Omega^{+}) = \alpha$ ist.\\
        Wegen $\mu(\Omega^{+}) = \mu(C_n) + \mu(\Omega^{+} \setminus C_n)$ und $\mu(\Omega^{+} \setminus C_n) \subset \Omega^{+}$ folgt $\mu(\Omega^{+} \setminus C_n) \geq 0$.
        Folglich erhalten wir $\mu(\Omega^{+}) = \mu(C_n) + \mu(\Omega^{+} \setminus C_n) \geq \mu(C_n)$. 
        Daraus folgt $\mu(\Omega^{+}) \geq \mu(C_n) \geq \mu(B_n) \overset{n \to \infty}{\longrightarrow} \alpha$.
        Dann ist $\mu(\Omega^{+}) = \alpha$ gezeigt, und ist $\mu$ ein endliches Maß, und somit auch $\alpha < \infty$ gezeigt.\\
        \textit{Schritt 3}. Es sei $\Omega^{-} = \Omega \setminus \Omega^{+}$, und zeigen wir, dass $\mu(B) \leq 0$, für jedem $B \in (\Omega^{-} \cap \mathcal{A})$. 
        Angenommen, wäre für eine $B \in (\Omega^{-} \cup \mathcal{A})$ mit $\mu(B) > 0$. Wir haben festgestellt, dass $\Omega^{+} \cap B = \emptyset$. Dann wäre
        \[\mu(\Omega^{+} \cup B) = \mu(\Omega^{+}) + \mu(B) = \alpha + \mu(B) > \alpha.\]
        Widerspruch zur Definition von $\alpha$. Also gilt $\mu(B) \leq 0$ für alle $B \in (\Omega^{-} \cup \mathcal{A})$. Das ist (b).
    \end{proof}
    
    \begin{corollary}\label{korollar:1}
        Jedes endliche signierte Maß $\mu$ auf $(\Omega, \mathcal{A})$ ist Differenz zwei endlicher Maße.
    \end{corollary}
    
    \begin{proof}
        Es sei $\Omega = \Omega^{+} \cup \Omega^{-}$ eine Hahn-Zerlegung von \cref{satz:1}. Dann werden durch 
        \[\mu^{+}(A) = \mu(A \cap \Omega^{+}), \quad \mu^{-}(A) = - \mu(A \cap \Omega^{-})\]
        offenbar Maße für alle $A \in \mathcal{A}$ definiert, und es gilt $\mu = \mu^{+} - \mu^{-}$, wegen 
        \[A = (A \cap \Omega^{+}) \cup (A \cap \Omega^{-}).\]
    \end{proof}

    Dagegen ist die Hahn-Zerlegung nicht eindeutig.
    \begin{corollary}
        Sei $\mu$ ein endliches signiertes Maß auf $(\Omega, \mathcal{A})$. Seien $\Omega = P_1 \cup N_1 = P_2 \cup N_2$ 
        zwei Hahn-Zerlegungen von $\Omega$ bezüglich $\mu$. Dann gelten symmetrische Differenz
        von $P_1, P_2$ und $N_1, N_2$ Nullmengen sind. 
    \end{corollary} 

    \begin{proof}
        Es gilt 
        \[P_1 \Delta P_2 = (P_1 \setminus P_2) \cup (P_2 \setminus P_1).\]
        Dann sind $P_1 \setminus P_2 = P_1 \cap {P_2}^{\mathrm{c}}$ ist eine Teilmenge der positiven Menge $P_1$, und eine Teilmenge der negativen Menge ${P_1}^{\mathrm{c}}$. 
        Also ist $P_1 \setminus P_2$ Nullmenge, und analog ist $P_2 \setminus P_1$ Nullmenge.  
        Dann gilt $\mu(P_1 \Delta P_2) =0$.
    \end{proof}

    Um eine eindeutige Zerlegung zu erhalten, fugt man eine zusätzliche Bedingung hinzu.
    \begin{definition}
        Zwei signierte Maße $v_1, v_2$ auf $(\Omega, \mathcal{A})$ heißen \textit{zueinander singulär}, und man schreibt dafür $v_1 \bot v_2$, 
        wenn es eine Menge $A \in \mathcal{A}$ gibt mit $v_1(A)=v_2(A^{\mathrm{c}})=0$.
    \end{definition}

    \begin{mdframed}[style=mdfexample]
        \begin{corollary}[Zerlegung von Jordan]\label{satz:3}
            Sei $\mu$ ein signiertes Maß auf $(\Omega, \mathcal{A})$. Es gibt genau eine Zerlegung $\mu = \mu^{+} - \mu^{-}$ mit zueinander singulären 
            Maßen $\mu^+, \mu^-$.
        \end{corollary}
    \end{mdframed}

    \begin{proof}
        Seien $(P, N)$ eine Hahn-Zerlegung im Sinne von \cref{satz:1}, und für alle $A \in \mathcal{A}$ existieren 
        \[\mu^{+}(A) = \mu(A \cap P)\]
        und 
        \[\mu^{-}(A) = -\mu(A \cap N).\]
        Aus \cref{korollar:1} folgt $\mu = \mu^{+} - \mu^{-}$ und $\mu^{+}, \mu^{-}$ sind zwei endliche nichtnegative Maße. Dann gilt $\mu^{+}(N) = \mu^{-}(P) = 0$, d.h. $\mu^{+} \bot \mu^{-}$. 
        Dies zeigt die Existenz.\\
        Seien $(\tilde{P}, \tilde{N})$ eine weitere Hahn Zerlegung bezüglich $\mu$, für jedem $A \in \mathcal{A}$ mit $A \subset \tilde{P}$ gilt 
        \[\mu(A) = \mu^{+}(A) - \mu^{-}(A) = \mu^{+}(A).\]
        Folglich erhalten man 
        \[\mu^{+}(A) = \mu^{+}(A \cap \tilde{P})  + \mu^{+}(A \cap \tilde{N}) = \mu^{+}(A\ \cap \tilde{P}).\]
        Analog ist 
        \[\mu^{-}(A) = -\mu^{-}(A \cap \tilde{N}),\]
        so dass die Eindeutigkeit von Jordan-Zerlegung $(\mu^{+}, \mu^{-})$ bewiesen ist. 
    \end{proof}
\end{document}