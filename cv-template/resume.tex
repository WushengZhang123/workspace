\documentclass{resume}
\usepackage{zh_CN-Adobefonts_external} 
\usepackage{linespacing_fix}
\usepackage{cite}
\begin{document}
\pagenumbering{gobble}



%***"%"后面的所有内容是注释而非代码,不会输出到最后的PDF中
%***使用本模板,只需要参照输出的PDF,在本文档的相应位置做简单替换即可
%***修改之后,输出更新后的PDF,只需要点击Overleaf中的“Recompile”按钮即可
%**********************************姓名********************************************
\name{方鸿渐}
%**********************************联系信息****************************************
%第一个括号里写手机号,第二个写邮箱
\contactInfo{(+86) 1234567890}{test@test.test}
%**********************************其他信息****************************************
%在大括号内填写其他信息,最多填写4个,但是如果选择不填信息,
%那么大括号必须空着不写,而不能删除大括号。
%\otherInfo后面的四个大括号里的所有信息都会在一行输出
%如果想要写两行,那就用两次这个指令(\otherInfo{}{}{}{})即可
\otherInfo{性别:男}{籍贯:江南}{}{}
\otherInfo{来历:钱钟书《围城》}{}{}{}
%*********************************照片**********************************************
%照片需要放到images文件夹下,名字必须是you.jpg,如果不需要照片可以不添加此行命令
%0.15的意思是,照片的宽度是页面宽度的0.15倍,调整大小,避免遮挡文字
\yourphoto{0.15}
%**********************************正文**********************************************


%***大标题,下面有横线做分割
%***一般的标题有:教育背景,实习(项目)经历,工作经历,自我评价,求职意向,等等
\section{教育背景}


%***********一行子标题**************
%***第一个大括号里的内容向左对齐,第二个大括号里的内容向右对齐
%***\textbf{}括号里的字是粗体,\textit{}括号里的字是斜体
\datedsubsection{\textbf{克莱登大学},克莱登实验班,\textit{博士}}{1926.09 - 1930.06}


%***********列举*********************
%***可添加多个\item,得到多个列举项,类似的也可以用\textbf{}、\textit{}做强调
\begin{itemize} [parsep=1ex]
  \item \textbf{证书来源}:购买自爱尔兰商人
\end{itemize}


\datedsubsection{\textbf{北平某大学},实验班,\textit{本科}}{1922.09 - 1926.06}
\begin{itemize} [parsep=1ex]
  \item \textbf{土木工程系}:不喜欢,转系
  \item \textbf{社会学系}:不喜欢,转系
  \item \textbf{中国文学系}:从此毕业
\end{itemize}

\section{职业经历}

\datedsubsection{\textbf{点金银行},职员}{1930.09}
\begin{itemize}[parsep=0.5ex]
  \item 高中订婚的未婚妻的父亲的公司
\end{itemize}

\datedsubsection{\textbf{三闾大学},副教授}{1931.09 - 1934.6}
\begin{itemize}[parsep=0.5ex]
  \item 同事:李梅亭、顾尔谦、孙柔嘉、赵辛楣
\end{itemize}

\datedsubsection{\textbf{上海某报社},职员}{1934.09 -}
\begin{itemize}[parsep=0.5ex]
  \item 生活不如意
\end{itemize}

\section{情感经历}

\datedsubsection{\textbf{鲍小姐},一夜情}{1930.06}
\begin{itemize}[parsep=0.5ex]
  \item \textbf{地点}:回国船上
\end{itemize}

\datedsubsection{\textbf{苏文纨},单恋}{1930.08 - 1931.08}
\begin{itemize}[parsep=0.5ex]
  \item \textbf{地点}:上海
  \item 苏小姐一直钟情于方鸿渐,也错以为方鸿渐对其有意
  \item 方鸿渐在月夜下情境所迫,吻了苏小姐,第二日不得不告诉苏小姐自己所爱并非她
\end{itemize}

\datedsubsection{\textbf{唐晓芙},热恋}{1930.08 - 1931.08}
\begin{itemize}[parsep=0.5ex]
  \item \textbf{地点}:上海
  \item 闲暇之余经常去苏小姐家探望,由此结识了苏小姐的表妹唐晓芙唐小姐
  \item 由于苏小姐挑拨,唐小姐也与方鸿渐一刀两断
\end{itemize}

\datedsubsection{\textbf{孙柔嘉},妻子}{1934.08 -}
\begin{itemize}[parsep=0.5ex]
  \item \textbf{订婚地点}:三闾大学
  \item 方、孙二人在赵辛楣家中遇上苏文纨,神情谈吐间遭到苏小姐的讽刺
  \item 二人回到上海,因工作、父母、亲戚妯娌等多方面问题又多次激发矛盾
\end{itemize}

\section{简历写作注意事项}

写作时不要泛泛而谈太笼统,要应用STAR原则,即Situation(情景)、Task(任务)、Action(行动)和Result(结果)四个英文单词的首字母组合。

\begin{itemize}[parsep=0.5ex]
  \item S指的是situation,事情是在什么情况下发生
  \item T指的是task,你是如何明确你的目标的
  \item A指的是action,针对这样的情况分析,你采用了什么行动方式
  \item R指的是result,结果怎样,在这样的情况下你学习到了什么
\end{itemize}

\end{document}
