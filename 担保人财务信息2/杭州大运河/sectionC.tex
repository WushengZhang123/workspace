\documentclass[H:\workspace\担保人财务信息2\杭州大运河\HangZhouText.tex]{subfiles}
\begin{document}
\section{“水韵江南,盘门夜秀”夜游活动}
我们可以先看一下这个项目,水城旅游。
其中,“水韵江南,盘门夜秀”夜游活动\footnote{c.f. \href{http://www.zbwmy.com/150/154/2021/10/08/32129365.html}{招标公告}} 值得我们关注。\par 
苏州文化旅游发展集团有限公司(以下简称“公司”)关于水上旅游项目的规划与建设主要由苏州东方水城旅游发展有限公司
与苏州市文化广电和旅游局负责。\par 
\subsection{关于苏州东方水城旅游发展有限公司}
苏州东方水城旅游发展有限公司成立于 2007 年 6 月 4 日,注册资本:人民
币 3,000.00 万元,注册地址:苏州市桃花坞龙兴桥 33 号。法定代表:韩惊雷。
经营范围:市区旅游运输;停车场管理(限自有物业);中餐制售;境内旅游业
务;入境旅游业务;出境旅游业务;环城河旅游项目的投资、建设、开发、经营;
旅游咨询、旅游策划、会务接待;设计、制作、代理户外广告,利用自有媒介发
布户外广告;房屋租赁;销售:百货、工艺美术品。(依法须经批准的项目,经
相关部门批准后方可开展经营活动)。\par 
截至 2021 年 3 月 31 日,苏州东方水城旅游发展有限公司未经审计的资产总
额为 10.53 亿元,负债总额为 6.57 亿元,所有者权益为 3.96 亿元。2021 年 1-3
月,苏州东方水城旅游发展有限公司未经审计实现的营业收入为 830.00 万元,
实现的净利润为-291.00 万元。苏州东方水城旅游发展有限公司亏损的主要原因
是下属控股公司游船计提减值准备以及管理费用较高。
    
\subsection{项目介绍}
“水韵江南·盘门夜秀”夜游项目活动紧紧围绕运河文化旅游主题,通过城墙灯光秀门、创意盘市集等主题活动打造蕴涵苏州2500年
历史韵味的大运河文化旅游新场景、夜游产品。投资达620万元。\footnote{c.f. \href{http://czju.suzhou.gov.cn/zfcg/html/project/1b96750caa1b4973bacff49e3935a330.shtml}{招标公告}}
同年8月,苏州东方水城旅游发展有限公司受苏州市文化广电和旅游局委托,负责“水韵江南 盘门夜秀”夜游活动的开发、建设,
后期运营中将在景区内组织园林时尚风等配套活动以及旅游线路产品,从文商旅相结合
角度展示运河文化和文旅创新,充分发挥苏州古城文化底蕴的优势。\par 
由于该项目是今年8月新提出的,所以一切都还不完善。因此我们可以看看下面几个项目。
    
\section{环古城河水上观光游项目}
2016 年该公司下属全资子公司东方水城与苏汽集团合资组建了苏州水上旅
游发展有限公司(简称“水上游公司”),运用市场化手段开展环古城河水上旅
游整合工作,截至 2021 年 3 月末,共有在运营游船 46 艘,客位 2395 个,种类
包括各类画舫船、欧式游船、双层游艇等,可满足不同类型客户的观光需求,
游船配备了沿途讲解及评弹演出等,同时推出了餐饮、婚礼包船等业务,并开
通了畅游苏州、水上巴士等线路,研发了美食专线等新产品。2019 年以来,受
一日游整顿\footnote{2018 年 10 月起,苏州市相关部门联合执法打击“非法一日游”,查封多家违法违规旅行社及服
务网点,严查购物店涉嫌逃税漏税、价格欺诈、商业贿赂、销售假冒伪劣商品等违法行为,进一
步完善各景区、水上游、购物场所和旅行社一日游产品的价格公示制度,加强价格监管。}
影响,团体游客总数量明显减少,当年游客数量 100 万人次,较上
年下降 61.09\%,当年实现营业收入 0.42 亿元,同比下降 24.86\%。\par 
    
2020 年以来随着新冠疫情的爆发,该公司旅游及景区运营板块受到较大的
冲击。其中环古城河水上观光游项目疫情期间暂停运营,3 月底开始恢复运营,2020 年 2 月和 3 月
的(预计)营业收入均为 0,而去年同期的收入分别为 272 万元和 339 万元。
2020 年全年公司旅游及景区运营板块收入较原预算数预计减少 3,000 万元,当年游客数量 52.8 万人次,较上年
下降 47.2\%,当年实现营业收入 0.2 亿元,较上年下降 54.70\%。\par 
    
2020 年以来为应对疫情影响,水上游公司先后开发了环古城河畅游线、环古
城河尊享游等一系列新产品,并对环古城河南线灯光进行提升,与其他单位合
作开发姑苏夜画北线项目。\footnote{c.f. 
\href{http://file.finance.sina.com.cn/211.154.219.97:9494/MRGG/BOND/2020/2020-7/2020-07-13/14855437.PDF}{苏州文化旅游发展集团有限
公司主体信用评级报告} p. 14 - 15}
    
\subsection{资金情况}
该公司作为承担苏州市文化旅游资源整合和文化旅游项目开
发建设的重要国有企业集团,在当地的经营优势明显,得到当地政府在政策
扶持、资金安排、资源配置等方面的大力支持。2016-2018 年及 2019 年第
一季度公司获得政府补助分别为 0.11 亿元、0.04 亿元、0.16 亿元及 0.46 亿元,拆迁补偿分别为 0、0.007 亿元、0.03 亿元
及 0.25 亿元。同期末公司专项应付款余额分别为 2.03 亿元、2.00
亿元、2.20 亿元及 2.22 亿元。此外,公司为实现金融资本与文旅产业的融合发展,未来仍会
加大金融类企业股权投资支出。该公司主要通过银行借款、发行债券进行融资,此外股东增资也为公司
带来较多资金补充。2016-2018 年及 2019 年第一季度公司筹资环节产生的现
金流量净额分别为 4.61 亿元、11.29 亿元、-2.20 亿元及 1.08 亿元,其中 2017
年因获得股东增资 6 亿元增长较多,2018 年公司偿还了较多银行借款。\footnote{c.f.
\href{http://file.finance.sina.com.cn/211.154.219.97:9494/MRGG/BOND/2020/2020-7/2020-07-13/14855437.PDF}{公司主体信用评级报告} p.28}

截至 2019 年 3 月末,该公司主要在建项目计划总投资 40.63 亿元,已累计
投资 16.99 亿元,已回笼资金 12.14 亿元。其中“两河一江”环古城河提升、
古城墙保护修缮二期工程、环古城河慢行系统、环古城河健身步道及提升项目、
码头改造、城湖联动等公益性项目已完工,已累计投入 3.44 亿元,经财政评
审后,截至 2019 年 3 月末累计收到市财政给予补助 2.71 亿元。\footnote{c.f. 
\href{http://file.finance.sina.com.cn/211.154.219.97:9494/MRGG/BOND/2020/2020-7/2020-07-13/14855437.PDF}{苏州文化旅游发展集团有限公司主体信用评级报告} p.17}\par 
此外,公司还拟对游船更新改造项目投资,计划总投资 2.50
亿元,主要包括改造 7 艘欧Ⅴ动力大中型游船、新建 16 艘 60 客位混合动力大
型游船、25 艘 45 客位混合动力中型游船、15 艘 20 客位纯电动小型游船,以
及岸上充电装置等配套设施等,截至 2019 年 3 月末已投入 0.65 亿元。\footnote{c.f. \href{http://file.finance.sina.com.cn/211.154.219.97:9494/MRGG/BOND/2020/2020-7/2020-07-13/14855437.PDF}{苏州文化旅游发展集团有限公司主体信用评级报告} p.18}
\end{document}