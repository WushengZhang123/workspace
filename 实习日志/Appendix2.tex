\documentclass[H:\workspace\实习日志\ShixiRizhi.tex]{subfiles}
\begin{document}
\section{关于数据库的一些认识}\label{appendix:b}
\subsection{背景介绍}
业务部每天查询、调用大量数据,有些是很常用的,比如某市的经济指标、人口,或者全国城投的评级,等等,
但我们做业务的,每次都是专用专取,缺乏常用数据的保存与分组,这样不仅会因为“重复劳动”而效率低下,
另外,不同的数据来源很容易导致查询到的数据缺乏真实性,而验证数据会添加更多的不确定性因素。还有一点,
由于我们以往做过的业务太多,缺少对已完成和、正在做的项目的统计。我不是让大家增加负担,但缺少已完成的项目的信息,
当我们面对一个新的项目时,我们不确定是否已经做过,也就意味着,不确定是否有可以借鉴的资料,
而对于正在做的项目,我们会容易遗忘许多细节,比如担保公司的名称、项目进度,等等。小的细节不关注,会导致更多的麻烦,
学会整理并保存数据,是我们上学时学的第一课,事实证明,这很有意义。因此,基于以上几个原因,我决定为组里建立一个简单可用、
容易维护的数据库。\par 
数据库于2021年11月16日开始搭建,经过将近两个星期的摸索与学习,我已经掌握了使用 \textbf{MySQL} 与 \textbf{Oracle} 的方法,
经过思考并询问一些程序员,我使用 MySQL 搭建数据库,原因有三: 
\begin{itemize}
    \item MySQL 是免费、开源的,而 Oracle 要收费;
    \item MySQL 目前使用人数最多,有许多模板、资料可以借鉴、学习;
    \item 相对 Oracle,MySQL 更容易学习与维护。
\end{itemize} 

\subsection{数据库介绍}
已建立江苏省区县数据库,包含一个主表,数据,和几个子表,包含
\begin{itemize}
    \item cxustomerinfo\footnote{一些已有的资料汇总};
    \item 全国城投评级;
    \item (国家示范)农业园;
    \item 国家示范产业园;
    \item (江苏省各个区县)基本信息;
    \item 小镇;
    \item 已投放客户;
    \item (江苏省各个区县)数据;
    \item (数据)汇总;
    \item 物流园;
    \item 省级示范园;
    \item 示范县;
    \item 示范点;
\end{itemize}

已建立 登记详细信息列表,包含一个主表,应收账款质押,和三个子表,
\begin{itemize}
    \item 应收账款{\_}质权人;
    \item 应收账款质押;
    \item 应收账款质押{\_}出质人;
    \item 应收账款质押{\_}合同;
\end{itemize}

\subsection{未完成的项目}
对于\uwave{登记详细信息列表},计划采用雪花结构,需要对应收账款、应收账款(保理)和
融资租赁三表进行分表操作,每个表再新建子表,包含质权人、出质人与合同,
用\uwave{应收账款}中的 “序号” 做 foreign key,让这三个表产生联系。再新建立一个 fact table,保留 “序号”。\par 

还需要再建立一个新数据库,用来存放财务数据。采用 star schema,也就是说,fact table 保留 
“资产负债表{\_}id”,“利润表{\_}id”,“现金流表{\_}id”,“指标{\_}id”。dimension tables 包括\uwave{资产负债表}、
\uwave{利润表}、\uwave{现金流表}与\uwave{指标}。每个表都建立一个 id 做主键,与 fact table 匹配。
\uwave{资产负债表}至少应包含列 “流动资产总计”、“非流动资产总计”、“流动负债总计”、“非流动负债总计” 与 “所有者权益总计”,还有 “时间”。
\uwave{利润表}至少应包含列 “总收入”、“总成本”、“利润总额” 与 “时间”。\uwave{现金流表}至少应包含 “经营活动的现金流量
净额”、“投资活动的现金流量净额” 与 “筹资活动产生的现金流量净额” 与 “时间"。\uwave{指标}应当对不同的财务指标
进行分类,至少应该包含 “偿债能力”、“盈利能力” 和 “运营能力" 与 “时间”。\par 

不需要将财务数据按年份分别建表,连表查询的代码量会很大。\par 

关于表的设计与维护,可以参照 \href{https://developer.aliyun.com/topic/java20}{阿里巴巴Java开发手册-2021最新嵩山版},
其中有对于 MySQL的写法说明。如果你有更好的设计,也可以尝试一下,不要在这个数据库的建立与维护上不花心思,
这个是值得的。目前,这家公司没有相应的氛围,如果没有人支持,也可以放弃,在愤怒时,可以输入以下代码
\begin{lstlisting}[language=C,style=mystyle]
    sudo rm -rf/ 
\end{lstlisting}

然后开始新的生活。
最后,我是要跑路了,祝后来者身体健康。

\subsection{一些关于 MySQL 的认识}
这一部分是为了写给那些没有 SQL 基础的人,让他们对查询数据有一个基本的认识。\par 

首先是连接 MySQL 服务器。
\lstset{style=mystyle,language=SQL}
\begin{lstlisting}
    mysql -u root -p
    Enter password: hwrz@8888 
\end{lstlisting}

如果登录成功,会出现 \colorbox{lightgray}{mysql>} 命令提示窗口,你可以在上面执行任何 SQL 语句。
\begin{lstlisting}
    Welcome to the MySQL monitor.  Commands end with ; or \g.
    Your MySQL connection id is 9
    Server version: 8.0.27 MySQL Community Server - GPL
    
    Copyright (c) 2000, 2021, Oracle and/or its affiliates.
    
    Oracle is a registered trademark of Oracle Corporation and/or its
    affiliates. Other names may be trademarks of their respective
    owners.
    
    Type 'help;' or '\h' for help. Type '\c' to clear the current input statement.
    
    mysql> select column /* 需要查询的列 */
    from table /* 需要查询的表 */
    where column = ... /* 额外的限制条件 */
    order by column asc/desc; /* 对某一列进行排序,正序用 asc,倒序用 desc */
\end{lstlisting}

如果想查询两个表中的数据,可以用以下命令
\begin{lstlisting}
    /* 选择需要查询的数据 */
    select col1, col2 from table a
    /* 
    用 join 语句联合两个表 */
    join table b
    /* 
    让两张表中的某列数据相联 */
    on a.id = b.id
    /* 
    如果有更进一步的限制 */
    where columnname = ...
    /* 
    如果想对查询数据进行排序 */
    order by columnname asc/desc;
\end{lstlisting}
联表的方式有许多种,join 语句指的是同时历遍两张表的所有数据。
\end{document}