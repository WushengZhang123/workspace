\documentclass[H:\workspace\实习日志\ShixiRizhi.tex]{subfiles}
\begin{document}
\section{部分日志}\label{appendix:a}
\subsection{10月8日}
时间:10:00 \textasciitilde \, 6:12\\
工作安排:
\begin{itemize}
    \item 熟悉公司人事与管理制度;
    \item 得到《内部管理制度》一书,学习行政管理制度;
    \item 录入宿州现代农业投资集团与宿州城投集团的财务数据;
    \item 参与公司的例会;
\end{itemize}
自我评价:尽快熟悉公司的各种制度,思考下一步需要准备什么材料,具备哪些能力。有一个想法,需要尽快复习
Wahrscheinlicheitstheorie与Brownsche Bewegung的相关内容。如有必要,尽快学会熟练使用SQL。尽快复习R的相关操作,深入学习
time series analysis的相关原理。可以试着在R上对Sonic Healthcare的股价与 Prof. Michael von Thaden提供的数据进行预测。

\subsection{10月9日}
时间:8:30 \textasciitilde \, 6:30\\
工作安排:
\begin{itemize}
    \item 熟悉公司的考勤制度;
    \item 复习time series的相关内容,练习在R中运用ARIMA Model预测未来数据;
    \item 复习hypothesis testing 与 linear regression的数学原理;
    \item 签实习生合同,从法律上明确实习生与公司的责任与义务;
    \item 读云台山项目与苏州科技项目的评审报告的评审报告,写摘要;
\end{itemize}

\subsection{10月11日}
工作安排:
\begin{itemize}
    \item 下载并熟悉R Studio其中的基本操作;
    \item 复习TikZ的基本操作,并练习几个流程图;
    \item 回顾metropolis包的官方文件;
    \item 参与公司的例会;
    \item 复习Stochastik的基础知识,尤其是有关 Martingale和 Brownsche Bewegung。
\end{itemize}
自我评价:找到一本有并投资租赁的书,学习业务部门的主要任务。在R Studio上练习对数据进行deskriptive Statistik, Hypothesentest, Zeitreiheanalysis, etc. 
复习metropolis包的设置,复习“跳转”是如何被设置的, i.e. beamergotobutton, hyperlink,以备后用。在例会中得知公司会发行ABS,我对此很感兴趣,我得去了解一下。

\subsection{10月12日}
工作安排:
\begin{itemize}
    \item 查找邳州、海陵区、高港区、新泰、滨州近三年的GDP与一般公共财政预算收入;
    \item 录入财务数据,计算常用financial ratios;
    \item 学习Teststatistik中几个基本原理,并在R Studio中练习Hypothesen与ARIMA Model的代码;
    \item 向同事询问了ABS发行的几个细节,i.e. 如何应对default risk,如何定价,对未来的预期,eic.
    \item 学会与同事相处,尽快融入新的生活。
\end{itemize}
自我评价:得知马哥负责定价,我对此很感兴趣,我要了解背后的统计学原理。Hypothesentest是应用统计学中常用的方法,ARIMA Model是我最用心学过的,只是我要了解方法背后的数学原理, ich muss, ich will... vincero ?

\subsection{10月13、14日}
工作安排:
\begin{itemize}
    \item 查找全江苏13个地级市的财政数据,i.e. GDP, 税收收入、政府性基金收入、一般公共预算收入, etc.
    \item 录入泰州市滨江开发有限公司的财务数据。
\end{itemize}

\subsection{10月20日}
工作安排:
\begin{itemize}
    \item 读马哥推荐的云台山项目与苏州科技项目的评审报告,并写摘要;
    \item 练习用TikZ制作流程图,学习如何制作长表格;
    \item 阅读 Werner A. Stahel所著的\textit{Statistische Datenanalyse}和 
    Robert H. Shumway与David S. Stoffer合著的\textit{Time Series Analysis},并在R上测试相应的代码。
\end{itemize}
自我评价:R 已入门,但关于使用map写循环还不太熟练,我需要知道更多有关Hypothesentest与Time Series Analysis的理论。
另外,需要更加深入地理解Numberische Analysis中关于Matrix Decomposition和Numerische Lösungen der Partiellgleichungen的理论。如果我有更多时间,需要学习SQL的相关理论。

\subsection{10月21日}
工作安排:
\begin{itemize}
    \item 读马哥推荐的评审报告,修改摘要;
    \item 读 Satz von Gauß-Markov 的应用与证明,在R中练习Regression的相关代码;
    \item 练习用TikZ制作流程图;
\end{itemize}

\subsection{10月22日}
工作安排:
\begin{itemize}
    \item 读马哥提供的评审报告,写完摘要;
    \item 读第四版的\textit{Time Series Analysis and Its Applications},
    找到一本书\textit{R for Everyone},在R上练习更多有关Time Series Analysis的操作。
\end{itemize}
点评:已经掌握base-r的基本操作,关于各种表格与ANOVA分析的代码要勤加练习,除此之外,没有什么太好的办法。
另外,已入门ARIMA Model的基础操作,需要更多的理论支持与练习。

\subsection{10月25日}
工作安排:
\begin{itemize}
    \item 读\textit{Time Series Analysis and Its Application},并在R上练习相应的代码;
    \item 读\textit{Variazanalyse}讲义,至Chochran定理,复习线性代数中关于Rank与Eigenvalue的知识。
\end{itemize}  
\end{document}